\subsection{Normals > Resolve direction}
\label{subsection:resolveNormalsDirection}

\index{normales}
\textcolor{red}{Cette fonction est une �bauche. Pour obtenir des normales sign�es, utilisez la m�thode \textit{Estimate Normals and Curvature} de la librairie PCL via le plugin qPCL (voir \ref{subsection:qPCL}).}\\

\par
Cette fonction tente de r�soudre le sens des normales d'un nuage de proche en proche, par propagation d'un ou plusieurs fronts sur le nuage (algorithme de type \emph{Fast Marching}).\\
\par
La propagation se fait sur une grille 3D (ici l'octree) et il faut donc choisir un niveau d'octree
auquel appliquer l'algorithme. Le choix du bon param�tre n'est malheureusement pas �vident, car un niveau
faible va entra�ner des cellules de taille importante, d'o� une propagation ais�e et rapide mais une mauvaise
prise en compte des circonvolutions locales, alors qu'un niveau �lev� va entra�ner l'inverse. De plus, plus
la propagation est difficile - i.e. par morceaux - plus le risque de voir des zones proches ayant des sens
oppos�s est forte. Il faut donc essayer l'algorithme � diff�rents niveaux d'octree, en commen�ant typiquement
� 5 ou 6, puis augmenter le niveau jusqu'� trouver une valeur satisfaisante.\\
\par
Note: la r�solution du sens des normales est au sens global pr�s, il peut donc �tre n�cessaire d'utiliser la fonction Invert\index{inversion} (Cf. section~\ref{subsection:invertNormals}) pour obtenir le r�sultat final recherch�.
