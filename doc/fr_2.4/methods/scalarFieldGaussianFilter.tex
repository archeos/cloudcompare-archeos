\subsection{Scalar Fields > Gaussian Filter}
\label{subsection:scalarFieldGaussianFilter}

\index{filtrage!gaussien}
\index{contraste!lissage|see{filtrage}}
\index{champ scalaire}

Application d'un filtre gaussien au champ scalaire actif.
\\
\par
L'utilisateur doit d�finir le noyau \emph{sigma} du filtre gaussien. Pour r�gler ce param�tre simplement, on peut se servir de l'octree\index{octree}, en prenant typiquement comme valeur la taille d'une cellule au niveau 8 pour un filtrage doux, 7 pour un filtrage relativement fort, etc. (la taille d'une cellule est affich�e au niveau de la console lorsqu'on affiche un \emph{rendu} du nuage via l'octree - Cf. section~\ref{subsection:octreeProp}).
\\
\par
Remarques :
\begin{itemize}
\item A partir de \emph{sigma}, on peut d�duire tr�s simplement le rayon de la sph�re en 3D d�limitant le voisinage qui sera consid�r� autour de chaque point. On calcule en effet pour chaque point la moyenne des valeurs scalaires de ses voisins, pond�r�e par la distance selon une loi gaussienne. Etant donn� que $3\sigma$ correspond � un �crasement du poids de 99,9\%, CloudCompare ne consid�re pas les points plus �loign�s.
\item Plus le noyau est grand, plus le calcul est lent.
\item Cette fonction est tr�s utile pour lisser le r�sultat d'un calcul du gradient\index{gradient} (section~\ref{subsection:scalarFieldGradient}) mais aussi d'un calcul de Portion de Ciel Visible (section~\ref{subsection:PCV}) sur un nuage de points par exemple.
\end{itemize}
