\subsection{Tools > Distances > Closest Point Set}
\label{subsection:closestPointSet}

Cette fonction calcule pour chaque point du nuage \emph{de r�f�rence}, le point le plus proche dans le nuage \emph{compared}.
L'ensemble de ces "points les plus proches" forme un nouveau nuage (\emph{Closest Point Set} ou CPS).
\\
\par
Remarques
\begin{itemize}
\item Pour appeler cette fonction, il faut s�lectionner exactement deux nuages de points.
\item On retrouve l'interface g�n�rique de choix du r�le\index{role@r�le} de chaque liste (Cf. section~\ref{subsection:chooseRole},
qui permet � l'utilisateur de pr�ciser quel est le nuage d'o� sont extraits les points du CPS (nuage \emph{compared})et quel est le nuage
de r�f�rence.
\item Le r�sultat est un nuage qui a exactement le m�me nombre de points que le nuage de r�f�rence, et dont 
chaque point appartient au nuage \emph{compared} (par d�finition). Par construction, il peut y avoir des doublons. 
C'est un r�sultat qui est utilis�, par exemple, par l'algorithme de recalage\index{recalage} entre deux nuages de 
points (Cf. section~\ref{subsection:register}).
%On peut aussi l'utiliser pour obtenir la partie d'un maillage "la plus proche" d'un nuage de point (en effet, lorsque le nuage
%est compos� des sommets d'un maillage, \emph{CloudCompare} segmente automatiquement le maillage en m�me temps que les sommets -
%ceci est un comportement g�n�rique).
\end{itemize}
