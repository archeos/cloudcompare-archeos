\subsection{Display > Toggle Centered Perspective}
\label{subsection:centeredPerspective}

\par
Dans le processus d'affichage, la projection\index{projection!pour visualisation} d�finit la mani�re dont les
objets 3D sont dessin�s � l'�cran de visualisation 2D.
L'affichage de \emph{CloudCompare} propose deux types de projections :
\begin{itemize}
\item parall�le orthographique : les points sont projet�s orthogonalement sur le plan image.
Le champ de vision correspond � un cylindre.
\item perspective\index{perspective|see{projection pour visualisation}} : les points sont projet�s vers un unique point n'appartenant
pas au plan image. Le champ de vision correspond � un c�ne.
\end{itemize}
Le plan image peut �tre assimil� � l'�cran de visualisation.
\\
\par
Dans \emph{CloudCompare}, la commande \emph{Toggle Centered Perspective} permet de basculer entre
l'affichage par projection orthographique, qui est le mode d'affichage par d�faut, et
l'affichage par projection perspective.
Lorsque cette commande est activ�e, le centre de rotation du point de vue est automatiquement
plac� sur le centre de la sc�ne observ�e. Lors des phases interactives (mouvement de souris
dans un contexte graphique - cf. section \ref{subsection:Interactivit�}), la cam�ra tournera donc
autour des objets de la sc�ne.
\\
\par
Si elle est sollicit�e une seconde fois, cette commande r�tablit l'affichage suivant une projection orthographique.\\
\par
\textcolor[rgb]{1.00,0.00,0.00}{Raccourci clavier : F3}
